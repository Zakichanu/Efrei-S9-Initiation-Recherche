%===============================================================================
% $Id: ifacconf.tex 19 2011-10-27 09:32:13Z jpuente $  
% Template for IFAC meeting papers
% Copyright (c) 2007-2008 International Federation of Automatic Control
%===============================================================================
\documentclass{ifacconf}

\usepackage{graphicx}      % include this line if your document contains figures
\usepackage{natbib}        % required for bibliography
%===============================================================================
\begin{document}
\begin{frontmatter}

\title{Sustainable development in banking, a lucrative trend or a gesture of humanity} 

\author[First]{Zakaria TOBBAL}
\author[2]{Yanis AMRAOUI}

\address[First]{Analyst Developer at Crédit Mutuel, and student for EFREI Paris, 30 Av. de la République, 94800 Villejuif (e-mail: zakaria.tobbal@efrei.net).}

\address[2]{Project Manager at Gireve, and student for EFREI Paris, 30 Av. de la République, 94800 Villejuif (e-mail: yanis.amraoui@efrei.net).}

\begin{abstract}                % Abstract of not more than 250 words.
Banking is such an important domain for our human being, in fact, banking are our best way to keep, take, track and give money toward society, because of global warming threat, banking needs to take the deep thinking of how to handle this.
\end{abstract}

\begin{keyword}
    Banking, Ecology, Energy, Cloud and Green Computing
\end{keyword}

\end{frontmatter}
%===============================================================================

\section{Introduction}
Banking companies begin, since the beginning of the 20's, to take the topic of sustainable development seriously.

Sustainability begin to become an subject trendy, in fact, with the influence of social media, companies, banking as well, try to make an advantage of it, influencing people, making a profit of it and in the end, not even making the world better.

Even if there is a lucrative trend around this subject, banking companies are, for a part of them, to make moves for sustainability. Indeed, we got some laws, strategies that are getting set-up by companies, region, nation organizations etc. The famous one is \textit{Corporate Social Responsibility} (CSR) and many more strategies. Our study will be based on CSR to get a clear idea of how this works.

\section{Corporate Social Reponsibility, a way to give credit to companies}


\subsection{CSR, what is it?}

 Corporate social responsibility (CSR) is a self-regulatory business model that assists a company in being socially accountable to itself, its stakeholders, and the general public. Companies that practice corporate social responsibility, also known as corporate citizenship, can be aware of the impact they have on all aspects of society, including the economic, social, and environmental.   CSR implies that a company operates in ways that benefit society and the environment rather than harming them in the ordinary course of business.

\subsection{Corporate CSR management}

Corporate social responsibility (CSR) is a self-regulatory business model that assists a company in being socially accountable to itself, its stakeholders, and the general public. Companies that practice corporate social responsibility, also known as corporate citizenship, can be aware of the impact they have on all aspects of society, including the economic, social, and environmental.   CSR implies that a company operates in ways that benefit society and the environment rather than harming them in the ordinary course of business.

\section{The green inflection}

\subsection{A resurgent debate}

Today the banks will have quite a few problems concerning the environment, whether it is the ability to make payments due to environmental costs, the laws their direct responsibility of the bank through management control for environmental damage or the reputation of the company Damage to it by association with polluting activities 

The debate over sustainability is nothing new (the Paris Agreement was signed in 2016). But today banks are feeling it from all sides. They are under scrutiny from the public, employees, associations and investors, and each group has slightly different motivations.

The pressure comes from the investor community and activist groups. Institutional investors have already submitted resolutions at their annual shareholder meetings, encouraging some banks to become more ambitious in the short and medium term to reduce their exposure to fossil fuel assets. With massive disruption looming, investors also expect banks to seek new growth and earnings opportunities.


\subsection{Sustainability Integration}

Banks began to consider sustainability in the 1990s, and interest in it increased in the 2000s. The banking industry now sees sustainability as an additional avenue for economic development. In practice, banks can promote sustainable development by directing their lending policies towards sustainable enterprises.  Initially, banks' social and environmental policies were implemented through 'environmental management' (the equivalent of 'risk management' in non-banking businesses), but this has been superseded by the current ESG approach. In fact, sustainability indicators and measures have been shown to improve the efficiency of environmental management systems and reward banks for better performance.

Banks are lagging behind other industries in participating in sustainability efforts. But Sustainability has only gained momentum in his banking and will continue to do so over the next decade.


\section{Conclusion}

It's that moment. Banks are feeling pressure from all sides and understand that climate change, if left unchecked, will have catastrophic environmental and macroeconomic consequences. Banks need more than just talking about green initiatives. They now have to prove they can meet stringent climate risk reporting standards.
Banks that plan and implement their sustainability agendas now will have a competitive edge in meeting their sustainability targets.

\begin{ack}
Place acknowledgments here.
\end{ack}

\bibliography{ifacconf}             % bib file to produce the bibliography
                                                     % with bibtex (preferred)
                                                   
%\begin{thebibliography}{xx}  % you can also add the bibliography by hand

%\bibitem[Abigael O.]{TowGreen:21}
%B.C. Abigael O., Amalia D., Constantinos M., Vasilios K.,
%\newblock Towards a Green Blockchain: A Review of Consensus Mechanisms and their Energy Consumption
%\newblock In A.F. Round, editor, \emph{Advances in Enzymology}, page
%  1. Smart Circular Economy/ Emerging Technology.

%\bibitem[Able et~al.(1954)Able, Tagg, and Rush]{AbTaRu:54}
%B.C. Able, R.A. Tagg, and M.~Rush.
%\newblock Enzyme-catalyzed cellular transanimations.
%\newblock In A.F. Round, editor, \emph{Advances in Enzymology}, volume~2, pages
%  125--247. Academic Press, New York, 3rd edition, 1954.

%\bibitem[Keohane(1958)]{Keo:58}
%R.~Keohane.
%\newblock \emph{Power and Interdependence: World Politics in Transitions}.
%\newblock Little, Brown \& Co., Boston, 1958.

%\bibitem[Powers(1985)]{Pow:85}
%T.~Powers.
%\newblock Is there a way out?
%\newblock \emph{Harpers}, pages 35--47, June 1985.

%\bibitem[Soukhanov(1992)]{Heritage:92}
%A.~H. Soukhanov, editor.
%\newblock \emph{{The American Heritage. Dictionary of the American Language}}.
%\newblock Houghton Mifflin Company, 1992.

%\end{thebibliography}

\end{document}